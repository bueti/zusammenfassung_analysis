\chapter{Differential- und Integralrechnung für Funktion von mehren Variablen}
\section{Funktion von mehreren Variablen}
\begin{definition}
Unter einer Funktion von zwei unabhängigen Variablen versteht man eine Vorschrift, die jedem geordneten Zahlenpaar $(x;y)$ aus einer Menge $D$ genau ein Element $z$ aus einer Menge $W$ zuordnet, Symbolische Schreibweise: $$z=f(x;y)$$
Analog gelangt man zu Funktionen von mehr als zwei unabhängigen Variablen:
$$u=f(x;y;z)$$
Es gibt vier verschiedene Darstellungsformen:
\begin{enumerate}
	\item \textit{Explizit} dargestellt:
	$$z = 2x + y + 1 \text{ oder } z = 2 \cdot sin(x-y)$$
	\item \textit{Implizit} dargestellt:
	$$ x^2 + y^2+z -1 = 0$$
	\item Dargestellt in einer \textit{Funktionstabelle}.
	\item Graphische Darstellung (als Fläche, Höhenlinien, Schnittkurven..).
\end{enumerate}
Die dritte und vierte Form eigenen sich nur für Funktionen mit maximal drei Variablen.
\end{definition}

\section{Grenzewert und Stetigkeit einer Funktion}
\subsection{Grenzwert}
\begin{definition}
Eine Funktion zweier Variablen hat an der Stelle $(x_,y_0)$ den Grenzwert $g$, wenn sich die Funkionswerte $f(x,y)$ beim Grenzübergang $(x,y) \rightarrow (x_0,y_0)$ unabhängig vom eingeschlagenen Weg dem Wert $g$ beliebig nähern. Symbolische Schreibweise:
$$\lim\limits_{(x,y) \rightarrow (x_0,y_0)} f(x,y) = g$$
Bemerkung:
\begin{enumerate}
\item Eine Funktion kann auch in einer Definitionslücke $(x_0,y_0)$ einen Grenzwert haben, obwohl sie dort nicht definiert ist.
\item Grenzwert auf einer Bildfläche: Je näher man zur Stelle $(x_0,y_0)$ kommt, desto flacher wird die Ebene auf der man sich bewegt.
\end{enumerate}
\end{definition}

\subsection{Stetigkeit}
\begin{definition}
Eine in $(x_0,y_0)$ und einer gewissen Umgebung von $(x_0,y_0)$ definierten Funktion $z = f(x;y)$ heisst an der Stelle $(x_0,y_0)$ \textit{stetig}, wenn der \textit{Grenzwert} der Funktion an der Stelle \textit{vorhanden} ist und mit dem dortigen Funktionswert \textit{übereinstimmt}.
$$\lim\limits_{(x,y) \rightarrow (x_0,y_0)} f(x,y) = f(x_0,y_0)$$
Anmerkung:
\begin{enumerate}
\item Die Stetigkeit an einer bestimmten Stelle setzt voraus, dass die Funktion dort auf \textit{definiert} ist. Ferner muss der Grenzwert an dieser Stelle existieren und mit dem Funktionswert übereinstimmen.
\item Eine Funktion $z = f(x,y)$ heisst dagegen an der Stelle $(x_0,y_0)$ \textit{unstetig}, wenn  $f(x_0,y_0)$ \textit{nicht} vorhanden ist oder $f(x_0,y_0)$ vom Grenzwert \textit{verschieden} ist oder dieser \textit{nicht} existiert.
\item Eine Funktion, die an \textit{jeder} Stelle ihres Definitionsbereichs stetig ist, wird als stetige Funktion bezeichnet.
\end{enumerate}
\end{definition}

\section{Partielle Differentiation}
\subsection{Partielle Ableitung 1. Ordnung}
\begin{definition}
Ist $z = f(x,y)$ an \textit{jeder} Stelle $(x,y)$ einer gewissen Bereiches \textit{partiell} differenzierbar, so sind die partiellen Ableitungen 1. Ordnung selbst wieder \textit{Funktionen} von $x$ und $y$. Wir definieren daher allgemein:
\begin{itemize}
\item Partielle Ableitung 1. Ordnung nach $x$:
$$ f_x(x;y) = \lim\limits_{\Delta x \rightarrow 0} \frac{f(x+\Delta x; y) - f(x;y)}{\Delta x}$$
\item Partielle Ableitung 1. Ordnung nach $y$:
$$ f_y(x;y) = \lim\limits_{\Delta y \rightarrow 0} \frac{f(x;y +\Delta y) - f(x;y)}{\Delta y}$$
\end{itemize}
Weitere übliche Symbole für partielle Ableitungen sind:
$$f_x(x;y) \text{, } z_x(x;y) \text{, } \frac{\partial f}{\partial x} (x;y) \text{, } \frac{\partial z}{\partial y} (x;y)$$
Es gelten die üblichen Ableitungsregeln aus dem ersten Semester.
\end{definition}

\subsection{Partielle Ableitungen höherer Ordnung}
\begin{definition}
Es wird genau gleich vorgegangen wie bei der Ableitung erster Ordnung. Die einzelnen Differentiationsschrite sind grundsätzlich in der Reihenfolge, in der die als Indizes angehängten Differentiationsvariablen im Ableitungssymbol auftreten, auszuführen. $f_{xy}$: Hier wird zunächst nach der Variablen $x$ und anschliessend nach der Variablen $y$ differenziert.
\end{definition}
\subsubsection{Satz von Schwarz}
\begin{definition}
Bei einer gemischten partiellen Ableitung $k$-ter Ordnung darf die Reihenfolge der einzelnen Differentiationsschritte vertauscht werden, wenn die partiellen Ableitungen $k$-ter Ordnung stetige Funktionen sind.
$$ f_{xy} = f_{yx} \text{, }  f_{xxy} = f_{yxx} \text{, usw...}$$
\end{definition}