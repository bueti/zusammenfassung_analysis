\chapter{Reelle Zahlenfolgen}

\section{Grenzwert/Limes}
\subsection{Definition}
\begin{itemize}
	\item Die reelle Zahl \(g\) heisst \textbf{Grenzwert} (Limes) der Zahlenfolge \(a_{n}\), wenn es zu jedem \(\in > 0\) eine natürliche Zahl \(n_{0} > 0\) gibt, so dass für alle \(n > n{0}\) stets gilt: \(|a_{n}-g| < \in \)
	\item Eine Folge \(a_{n}\) heisst \textbf{konvergent}, wenn sie einen Grenzwert \(g\) besitzt.
	\newline Symbolische Schreibweise: \(\lim\limits_{n \rightarrow \infty}{a_n} = g \)
\end{itemize}

\subsection{Stetigkeit einer Funktion}
Eine in \(x_0\) und einer gewissen Umgebung von \(x_0\) definierten Funktion \(y=f(x)\) heisst an der Stelle \(x_0\) stetig, wenn der Grenzwert der Funktion an dieser Stelle vorhanden ist und mit dem dortigen Funktionswert übereinstimmt:
\newline
\hspace*{10mm}\(\lim\limits_{x \rightarrow x_0}{f(x)} = f(x_0)\)
\newline Graphisch: Die Funktion macht keinen Sprung.

\subsection{Unstetigkeit}
Stellen in denen eine Funktion die Stetigkeitsbedingung \(\lim\limits_{x \rightarrow x_0}{f(x)} = f(x_0)\) nicht erfüllt ist, heissen Unstetigkeitsstellen.
\newline Eine Funktion \(f(x)\) ist also an der Stelle \(x_0\) unstetig, wenn mindestens einer der folgenden Aussagen zutrifft:
\begin{itemize}
	\item \(f(x)\) ist an der Stelle \(x_0\) nicht definiert
	\item Der Grenzwert von \(f(x)\) an der Stelle \(x_0\) ist nicht vorhanden
	\item Funktions- und Grenzwert sind zwar vorhanden, jedoch voneinander verschieden
\end{itemize}