\chapter{Funktionen und Kurven}
\section{Grundlagen}
\subsection{Definition}
Unter einer Funktion versteht man eine Vorschrift, die jedem Element \(x\) aus einer Menge \(D\) genau ein Element \(y\) aus einer Menge \(M\) zuordnet.

\subsection{Darstellung}
Symbolische Schreibweise
\newline \(y = f(x)\) mit \(x \in D\) oder
\newline \(f: x \rightarrow y = f(x)\) mit \(x \in D\)

\subsection{Analytische Schreibweise}
Explizite Darstellung: \(y = f (x) \)
\newline Implizierte Darstellung: \(F(x;y) = 0\)

\subsection{Nullstellen} % (fold)
\label{sec:nullstellen}
Eine Funktion \(y = f(x)\) besitzt an der Stelle \(x_{0}\) eine Nullstelle, wenn \(f(x_{0}) = 0\) ist.
\newline In einer Nullstelle schneidet oder berührt die Funktionskurve die x-Achse!

\section{Symmetrie} % (fold)
\subsection{Gerade Funktionen} % (fold)
Eine Funktion \(y = f(x)\) mit einem zum Nullpunkt symmetrischen Definitionsbereich D heisst gerade, wenn für jedes \(x \in D\) gilt:
\newline \(f(-x) = f(x)\)

\subsection{Ungerade Funktionen}
Eine Funktion \(y = f(x)\) mit einem zum Nullpunkt symmetrischen Definitionsbereich D heisst ungerade, wenn für jedes \(x \in D\) gilt:
\newline \(f(-x) = -f(x)\)

\subsection{Beweis}
Um zu Beweisen ob eine Funktione gerade oder ungerade ist, setzt man -x in die Funktion ein.

\subsection{Hornerschema}
Vorgehen:
\begin{enumerate}
	\item Erstes \(x\) muss geraten werden. Gute Kandidaten: \(\pm 1, \pm 2\). 
	\item In Tabelle einsetzen. Jede Reihe steht für ein \(x\), auch wenn \(x=0\) ist.
	\item Solange wiederholen bis alle \(x\) gefunden sind.
\end{enumerate}
Funktion: \( x^3-6x^2-4x^3+24\) \newline
\newline
\begin{tabular}{l|l|l|l|l}
	~ & \(1\) & \(-6\) & \(-4\) & \(24\) \\ \hline
	\(2\)& ~ & \(2\) & \(-8\) & \(-24\) \\ \hline
	~ & \(1\) & \(-4\) & \(-12\) & ~ \\ 
\end{tabular}

\section{Monotonie}
\(x_{1}\) und \(x_{2}\) seien zwei beliebige Werte aus dem Definitionsbereich \(D\) einer Funktion \(y = f(x)\) mit \(x_{1} < x_{2}\). Dann heisst die Funktion:
\begin{itemize}
	\item monoton wachsend, falls \(f(x_{1}) \ge f(x_{2}))\)
	\item streng monoton wachsend, falls \(f(x_{1}) < f(x_{2}))\)
	\item monoton fallend, falls \(f(x_{1}) \le f(x_{2}))\)
	\item streng monoton fallend, falls \(f(x_{1}) > f(x_{2}))\)
\end{itemize}

\section{Periodizität}
Eine Funktion \(y = f(x)\) heisst periodisch mit Periode \(p\), wenn mit jedem \(x \in D\) auch \(x \pm p\) zum Definitionsbereich \(D\) der Funktion gehört und es gilt:
\newline \(f(x \pm p) = f(x)\)

\section{Umkehr/inverse Funktion}
Eine Funktion \(y = f(x)\) heisst umkehrbar, wenn aus \(x_{1} \neq x_{2}\) stets \(f(x_{1}) \neq f(x_{2})\) folgt. Sie muss streng monoton sein!
\par
Der Definitions- und Wertebereich wird bei einer Umkehrung getauscht:
\newline \(f^{-1}(f(x)) = x = f(f^{-1}(x))\)
\newline Bestimmung:
\begin{enumerate}
	\item nach \(x\) auflösen
	\item \(x\) und \(y\) tauschen
\end{enumerate}
\section{Reelle Zahlenfolgen}
\subsection{Grenzwert/Limes}
\begin{itemize}
	\item Die reelle Zahl \(g\) heisst \textbf{Grenzwert} (Limes) der Zahlenfolge \(a_{n}\), wenn es zu jedem \(\in > 0\) eine natürliche Zahl \(n_{0} > 0\) gibt, so dass für alle \(n > n{0}\) stets gilt: \(|a_{n}-g| < \in \)
	\item Eine Folge \(a_{n}\) heisst \textbf{konvergent}, wenn sie einen Grenzwert \(g\) besitzt.
	\newline Symbolische Schreibweise: \(\lim\limits_{n \rightarrow \infty}{a_n} = g \)
\end{itemize}

\subsection{Stetigkeit einer Funktion}
Eine in \(x_0\) und einer gewissen Umgebung von \(x_0\) definierten Funktion \(y=f(x)\) heisst an der Stelle \(x_0\) stetig, wenn der Grenzwert der Funktion an dieser Stelle vorhanden ist und mit dem dortigen Funktionswert übereinstimmt:
\newline
\hspace*{10mm}\(\lim\limits_{x \rightarrow x_0}{f(x)} = f(x_0)\)
\newline Graphisch: Die Funktion macht keinen Sprung.

\subsection{Unstetigkeit}
Stellen in denen eine Funktion die Stetigkeitsbedingung \(\lim\limits_{x \rightarrow x_0}{f(x)} = f(x_0)\) nicht erfüllt ist, heissen Unstetigkeitsstellen.
\newline Eine Funktion \(f(x)\) ist also an der Stelle \(x_0\) unstetig, wenn mindestens einer der folgenden Aussagen zutrifft:
\begin{itemize}
	\item \(f(x)\) ist an der Stelle \(x_0\) nicht definiert
	\item Der Grenzwert von \(f(x)\) an der Stelle \(x_0\) ist nicht vorhanden
	\item Funktions- und Grenzwert sind zwar vorhanden, jedoch voneinander verschieden
\end{itemize}