\documentclass[a4paper,DIV10,12pt,headsepline,smallheadings,halfparskip-]{scrreprt}
\pagestyle{headings}
% Eine gute Codierung
\usepackage[utf8]{inputenc}
% Schriftart mit Umlauten
\usepackage[T1]{fontenc}
\usepackage{lmodern}
% Deutsche Beschriftung und Silbentrennung
\usepackage[ngerman]{babel}
% Für schöne Tabellen
\usepackage{longtable}
\usepackage{booktabs}
\usepackage{multirow}
% Um Grafiken zu laden
\usepackage{graphicx}
\usepackage{float}
% Klickbare Indexes und Kosmetik
\usepackage{color}
\definecolor{black}{gray}{0} % 10% gray
\usepackage[colorlinks=true,linkcolor=black,citecolor=black]{hyperref}
% Untersrecihungen
\usepackage[normalem]{ulem}
\setcounter{secnumdepth}{5}
\setcounter{tocdepth}{2}
% Mathe
\usepackage{amsmath}
\usepackage{mathtools}
% Dokument Infos
\title{Zusammenfassung Analysis I}
\author{Benjamin Bütikofer}

% Begin des eigentlichen Dokuments
\begin{document}
	\maketitle
	\tableofcontents

	\chapter{Funktionen und Kurven}
	\section{Darstellung} % (fold)
	\label{sec:darstellung}

	% section darstellung (end)
	\subsection{Definition}
	Unter einer Funktion versteht man eine Vorschrift, die jedem Element \(x\) aus einer Menge \(D\) genau ein Element \(y\) aus einer Menge \(M\) zuordnet.

	\subsection{Darstellung}
	Symbolische Schreibweise
	\newline \(y = f(x)\) mit \(x \in D\) oder
	\newline \(f: x \rightarrow y = f(x)\) mit \(x \in D\)

	\subsection{Analytische Schreibweise}
	Explizite Darstellung: \(y = f (x) \)
	\newline Implizierte Darstellung: \(F(x;y) = 0\)

	\subsection{Nullstellen} % (fold)
	\label{sec:nullstellen}
	Eine Funktion \(y = f(x)\) besitzt an der Stelle \(x_{0}\) eine Nullstelle, wenn \(f(x_{0}) = 0\) ist.
	\newline In einer Nullstelle schneidet oder berührt die Funktionskurve die x-Achse!

	\section{Symmetrie} % (fold)
	\subsection{Gerade Funktionen} % (fold)
	Eine Funktion \(y = f(x)\) mit einem zum Nullpunkt symmetrischen Definitionsbereich D heisst gerade, wenn für jedes \(x \in D\) gilt:
	\newline \(f(-x) = f(x)\)

	\subsection{Ungerade Funktionen}
	Eine Funktion \(y = f(x)\) mit einem zum Nullpunkt symmetrischen Definitionsbereich D heisst ungerade, wenn für jedes \(x \in D\) gilt:
	\newline \(f(-x) = -f(x)\)

	\subsection{Beweis}
	Um zu Beweisen ob eine Funktione gerade oder ungerade ist, setzt man -x in die Funktion ein.

	\section{Monotonie}
	\(x_{1}\) und \(x_{2}\) seien zwei beliebige Werte aus dem Definitionsbereich \(D\) einer Funktion \(y = f(x)\) mit \(x_{1} < x_{2}\). Dann heisst die Funktion:
	\begin{itemize}
		\item monoton wachsend, falls \(f(x_{1}) \ge f(x_{2}))\)
		\item streng monoton wachsend, falls \(f(x_{1}) < f(x_{2}))\)
		\item monoton fallend, falls \(f(x_{1}) \le f(x_{2}))\)
		\item streng monoton fallend, falls \(f(x_{1}) > f(x_{2}))\)
	\end{itemize}

	\section{Periodizität}
	Eine Funktion \(y = f(x)\) heisst periodisch mit Periode \(p\), wenn mit jedem \(x \in D\) auch \(x \pm p\) zum Definitionsbereich \(D\) der Funktion gehört und es gilt:
	\newline \(f(x \pm p) = f(x)\)

	\section{Umkehr/inverse Funktion}
	Eine Funktion \(y = f(x)\) heisst umkehrbar, wenn aus \(x_{1} \neq x_{2}\) stets \(f(x_{1}) \neq f(x_{2})\) folgt. Sie muss streng monoton sein!
	\par
	Der Definitions- und Wertebereich wird bei einer Umkehrung getauscht:
	\newline \(f^{-1}(f(x)) = x = f(f^{-1}(x))\)
	\newline Bestimmung:
	\begin{enumerate}
		\item nach \(x\) auflösen
		\item \(x\) und \(y\) tauschen
	\end{enumerate}

	\chapter{Reelle Zahlenfolgen}

	\section{Grenzwert/Limes}
	\subsection{Definition}
	\begin{itemize}
		\item Die reelle Zahl \(g\) heisst \textbf{Grenzwert} (Limes) der Zahlenfolge \(a_{n}\), wenn es zu jedem \(\in > 0\) eine natürliche Zahl \(n_{0} > 0\) gibt, so dass für alle \(n > n{0}\) stets gilt: \(|a_{n}-g| < \in \)
		\item Eine Folge \(a_{n}\) heisst \textbf{konvergent}, wenn sie einen Grenzwert \(g\) besitzt.
		\newline Symbolische Schreibweise: \(\lim\limits_{n \rightarrow \infty}{a_n} = g \)
	\end{itemize}

	\subsection{Stetigkeit einer Funktion}
	Eine in \(x_0\) und einer gewissen Umgebung von \(x_0\) definierten Funktion \(y=f(x)\) heisst an der Stelle \(x_0\) stetig, wenn der Grenzwert der Funktion an dieser Stelle vorhanden ist und mit dem dortigen Funktionswert übereinstimmt:
	\newline
	\hspace*{10mm}\(\lim\limits_{x \rightarrow x_0}{f(x)} = f(x_0)\)
	\newline Graphisch: Die Funktion macht keinen Sprung.

	\subsection{Unstetigkeit}
	Stellen in denen eine Funktion die Stetigkeitsbedingung \(\lim\limits_{x \rightarrow x_0}{f(x)} = f(x_0)\) nicht erfüllt ist, heissen Unstetigkeitsstellen.
	\newline Eine Funktion \(f(x)\) ist also an der Stelle \(x_0\) unstetig, wenn mindestens einer der folgenden Aussagen zutrifft:
	\begin{itemize}
		\item \(f(x)\) ist an der Stelle \(x_0\) nicht definiert
		\item Der Grenzwert von \(f(x)\) an der Stelle \(x_0\) ist nicht vorhanden
		\item Funktions- und Grenzwert sind zwar vorhanden, jedoch voneinander verschieden
	\end{itemize}

	\chapter{Differentialrechnung}
	\section{Differenzierbarkeit einer Funktion}
	Eine Funktion \(y=f(x)\) heisst an der Stelle \(x_0\) differenzierbar, wenn der Grenzwert

	\( \lim\limits_{\Delta x \rightarrow 0}{\frac{\Delta y}{\Delta x}} = \lim\limits_{\Delta x \rightarrow 0}{\frac{f(x_0 + \Delta x)-f(x_0)}{\Delta x}} = tan \alpha = m_{S(ekante)/T(angente)} \)

	definiert ist. Man bezeichnet ihn als die (erste) Ableitung von \(y=f(x)\) an der Stelle \(x_0\) oder als \textbf{Differentialquotient} von \(y=f(x)\) an der Stelle \(x_0\) und kennzeichnet ihn durch das Symbol:

	 \hspace*{10mm}\(y'(x_0)\), \(f'(x_0)\) oder \(\frac{dy}{dx} \mid_{x=x_0}\).

	Weitere nützliche Schreibweise:

	\hspace*{10mm}\(\frac{d}{dx}[f(x)] = f'(x)\)

	\begin{itemize}
		\item Die Differenzierbarkeit einer Funktion \(y=f(x)\) an der Stelle \(x_0\) bedeutet, dass die Funktionskurve an dieser Stelle eine eindeutig bestimmte Tangente mit endlicher Steigung besitzt.
		\item Die Ableitungsfunktion \(y'(x)=f'(x)\) ordnet jeder Stelle \(x\) aus einem Intervall \(I\) als funktionswert den Steigungswert (Grenzwert) zu. Man spricht dann kurz von der Ableitung einer Funktion \(y=f(x)\) an der Stelle \(x\)
	\end{itemize}
	Eine Betragsfunktion ist an der Stelle \(x=0\) nicht differenzierbar, da sie dort keine eindeutig bestimmte Tangente besitzt. Man muss also zu erst die rechts- und dann die linksseitige Ableitung ausrechnen.
	\par \textbf{Tangenten:} Für die Ableitung und den Steigungswinkel \(\alpha\) gilt \(y'=tan \alpha\)

	\section{Ableitungen der elementaren Funktionen}

	\begin{longtable}{|l|l|l|}
		\hline
		\multicolumn{2}{|l|}{Funktion f(x)}  & Ableitung f'(x) \\
		\hline
		Konstante Funktion & c = const & 0 \\
		\hline
		Potenzfunktion & \(x^n\) \newline \((x^2)^{-\frac{1}{2}} \) & \(n \cdot x^{n-1}\) \newline \(\frac{1}{(x^{2})^{\frac{1}{2}}}\) \\
		\hline
		Wurzelfunktion & \(\sqrt{x}\) gilt nur für 2. Wurzel & \(\frac{1}{2 \sqrt{x}}\) \\
		\hline
		\multirow{4}{*}{Trigonometrische Funktionen} & \(sin x\) \newline \(sin 2x\) & \(cos x\) \newline \(2 cos (2x)\) \\ \cline{2-3}
		& \(cos(x)\)  & \(-sin x\) \\ \cline{2-3}
		& \(tan(x)\) & \(\frac{1}{cos^2(x)}\), \( 1 + tan^2(x) \) \\ \cline{2-3}
		& \(cot(x)\) & \(-\frac{1}{sin^2(x)}\) \\
		\hline
		\multirow{4}{*}{Arkusfunktionen} & \(arcsin(x)\) & \(\frac{1}{\sqrt{1 - x^2}} \) \\ \cline{2-3}
		& \(arccos(x)\) & \(-\frac{1}{\sqrt{1-x^2}}\) \\ \cline{2-3}
		& \(arctan(x)\) & \(\frac{1}{1+x^2}\) \\ \cline{2-3}
		& \(arccot(x)\) & \(-\frac{1}{1+x^2}\) \\
		\hline
		\multirow{3}{*}{Exponentialfunktionen} & \(e^x\) & \( e^x \) \\ \cline{2-3}
		& \(e^{2x}\) & \( 2e^{2x} \) \\ \cline{2-3}
		& \(a^x\) & \((ln a) \cdot a^x \) \\ \hline
		\multirow{2}{*}{Logarithmusfunktionen} & \(ln x\) & \( \frac{1}{x} \) \\ \cline{2-3}
		& \(log_a x\) & \( \frac{1}{(ln a) \cdot x } \) \\ \hline
		\multicolumn{3}{|l|}{Hyperbelfunktionen, Areafunktionen: Siehe Seite 17 im Skript.} \\ \hline
	\end{longtable}

	\section{Ableitungsregeln}
	\subsection{Faktorregel}
	\(y = C \cdot f(x) \Rightarrow y' = C \cdot f'(x)\); C = Reelle Konstante, bleibt erhalten.
	\subsection{Summenregel}
	\(y = f_1(x)+f_2(x)+ ... +f_n(x) \Rightarrow y'=f'_1(x) + f'_2(x)+...+f'_n(x)\)

	Es darf gliedweise differenziert werden.
	\subsection{Produktregel}
	\(y=u \cdot v \Rightarrow y'=u' \cdot v + v' \cdot u\)

	Bei mehreren Termen:

	\(y=u \cdot v \cdot w \Rightarrow y'=u' \cdot v \cdot w + u \cdot v' \cdot w + u \cdot v \cdot w' \)
	\subsection{Quotientenregel}
	\(y = \frac{u}{v} \Rightarrow y' = \frac{u' \cdot v - v' \cdot u}{v^2}\)

	\section{Kettenregel}
	Überlegung: Was muss zuerst berechnet werde, was gehört zusammen?

	Beispiel:

	\hspace*{10mm} \(f(x) = arccos \sqrt{x^2-1} \)
	\newline
	\begin{tabular}{|l|l|}
		\hline
	   	innerste Funktion z & \(z: x \rightarrow z(x) = x^2-1\) \\
		\hline
		mittlere Funktion u & \(u: z \rightarrow u(z) = \sqrt{z}\) \\
		\hline
		äusserste Funktion F & \(F: u \rightarrow F(u) = arccos(u) \) \\
		\hline
	 \end{tabular}

	\chapter{Differentialrechnung}
	\section{Tangente und Normale}
	\begin{longtable}{p{0.5\textwidth}|p{0.5\textwidth}}
		Steigung der Tangente & \(m_T=f'(x)\)\\
		\hline
		Gleichung der Tangente & \(y_T = f'(x_0)(x-x_0) + y_0 \)\\
		\hline
		Steigung der Normalen & \(m_N=-\frac{1}{f'(x_0)} = \frac{y-y_0}{x-x_0} \)\\
		\hline
		Gleichung der Normalen & \(y_N = - \frac{1}{f'(x_0)}(x-x_0) + y_0 \) \\
	\end{longtable}

	\section{Linearisierung einer Funktion}
	In der Umgebung des Kurvenpunkts \(P=(x_0,y_0)\) kann die nichtlineare Funktion \(y=f(x)\) näherungsweise durch die lineare Funktion\newline
	\(y - y_0 = f'(x_0) \cdot (x - x_0) \) oder \( \Delta y = f'(x_0)\Delta x \)\newline
	ersetzt werden.
	
	\par Vorgehen: 
	\begin{enumerate}
		\item Arbeitspunkt P bestimmen: \(x_0, y_0\)
		\item Tangentensteigung:
		\begin{enumerate}
			\item erste Ableitung
			\item \(m_T = f'(x_0)\)
		\end{enumerate}
		\item Gleichung der Tangente in P
		\begin{enumerate}
			\item \(\frac{y-y_0}{x-x_0} = m_T  \Rightarrow y = 2x-3\pi  \)
		\end{enumerate}
		\item Im Arbeitspunkt P gibt linearisierte Funktion
		\item Exakter Wert: Ursprungsfunktion
		\item Näherungswert: Linearisierte Funktion
	\end{enumerate}

	\section{Monotonie einer Kurve} % (fold)
	\label{sec:monotonie_einer_kurve}
	\( y' = f'(x) > 0 \Rightarrow \) streng monoton wachsend\newline
	\( y' = f'(x) < 0 \Rightarrow \) streng monoton fallend
	% section monotonie_einer_kurve (end)

	\section{Krümmung einer Kurve} % (fold)
	\label{sec:krümmung_einer_kurve}
	\subsection{Allgemein} % (fold)
	\label{sub:allgemein}
	\( y'' = f''(x_0) > 0 \): Links-Krümmung. \( f'(x_0) \) muss streng monoton wachsend sein. \newline
	\( y'' = f''(x_0) < 0 \): Rechts-Krümmung. \( f'(x_0) \) muss streng monoton fallend sein. \newline
	Formel zur Berechnung der Krümmung einer ebenen Kurve \(y=f(x)\):\par
	\( \kappa = \kappa (x) = \frac{y''}{[1+(y')^2]^{\frac{3}{2}}} = \frac{f''(x)}{[1+[f'(x)]^2]^{\frac{3}{2}}} \)
	% subsection allgemein (end)
	% section krümmung_einer_kurve (end)
	\section{Charakterische Kurvenpunkte} % (fold)
	\label{sec:charakterische_kurvenpunkte}
	\subsection{Extremum} % (fold)
	\label{sub:extremum}
	Eine Funktion \(y = f(x) \) besitzt an der Stelle \(x_0\) ein relatives Maximum bzw. ein relatives Minimum, wenn in einer gewissen Umgebung von \(x_0\) stets \newline
	\(f(x_0) > f(x) \) bzw. \( f(x_0) < f(x) \) \newline
	ist (\(x \neq x_0\)).
		\[f'(x_0) = 0 \wedge f''(x_0) \neq 0 \]

	Für \(f''(x_0) > 0 \) liegt dabei eine relatives Minimum vor, für \(f''(x_0) < 0 \) dagegen ein relatives Maximum.
	% subsection extremum (end)
	\subsection{Wendepunkte, Sattelpunkte} % (fold)
	% \label{sub:wendepunkte}
	Punkte \(x_0\) mit \(f''(x_0) = 0 \) und \(f'''(x_0) \neq 0 \) heissen \textbf{Wendepunkte} (die Krümmung wechselt ihre Richtung).
	\par Kurven mit waagrechter Tangente werden als Sattelpunkte bezeichnet.
	\par Hinreichende Bedingung für einen \textbf{Wendepunkt}: \newline
	\[f''(x_0) = 0 \] und \[ f'''(x_0) \neq 0 \]
	\par Notwendige Bedingung für einen \textbf{Sattelpunkt}: \newline
	\[f'(x_0) = 0 \wedge f''(x_0) = 0 \wedge f'''(x_0) \neq 0 \]
	% subsection wendepunkte (end)

	% section charakterische_kurvenpunkte (end)
	\section{Kurvendisskusion}
	Ablauf:\newline
	\begin{itemize}
		\item Definitionsbereich/lücken
		\item Symmetrie (gerade, ungerade Funktion)
		\item Nullstellen, Schnittpunkt mit der y-Achse
		\item Pole, senkrechte Asymptoten (Polgeraden)
		\item Ableitungen (i. d. R. bis zur 3. Ordnung)
		\item Relative Extremwerte (Maxima, Minima)
		\item Wendepunkte, Sattelpunkte
		\item Verhalten der Funktion für \(x \rightarrow \pm \infty \), Asymptoten im Unendlichen
		\item Wertebereich der Funktion
		\item Zeichnung der Funktion in einem geeigneten Massstab
	\end{itemize}
	
	\chapter{Integralrechnung}
	\section{Einführung} % (fold)
	\label{sec:einführung}
	\subsection{Definition} % (fold)
	\label{sub:definition}
	Die Integration ist die Umkehrung der Ableitung.
	% subsection definition (end)
	\section{Stammfunktion} % (fold)
	\label{sub:stammfunktion}
	\begin{enumerate}
		\item Falls es eine Stammfunktion zu einer stetigen Funktion \(f(x)\) gibt, gibt es \emph{unendlich} viele Stammfunktionen
		\item Zwei beliebige Stammfunktionen \(F_1 (x)\) und \(F_2 (x)\) von \(f(x)\) unterscheiden sich durch eine \emph{additive} Konstante: \(F_1 (x) - F_2 (x) = C\)
		\item Ist \(F_1 (x)\) eine \emph{beliebige} Stammfunktion von \(f (x)\), so ist auch \(F_1 (x) + C\) eine Stammfunktion von \(f (x)\). Daher lässt sich die \emph{Menge aller Stammfunktionen} in der Form \(F(x) = F_1 (x) + C \) darstellen.
	\end{enumerate}
	% subsection stammfunktion (end)
	% section einführung (end)
	\section{Unter- und Obersumme}
	\[ \sum_{k=1}^{n} k = \frac{n(n+1)}{2} \]
	\[ \sum_{k=1}^{n} k^2 = \frac{n(n+1)(2n+1)}{6} \]

	\section{Das bestimmte Integral} % (fold)
	\label{sub:das_bestimmte_integral}
	Es gilt:\newline
	\[A = \lim\limits_{n \rightarrow \infty}U_n = \lim\limits_{n \rightarrow \infty}O_n = \lim\limits_{n \rightarrow \infty}\sum_{k=1}^{n} f(x_k) \Delta x_k = \int_a^b f(x)\,\mathrm{d}x \]
	% section das_bestimmte_integral (end)

	\section{Das unbestimmte Integral} % (fold)
	\label{sec:das_unbestimmte_integral}
	\[ I(x) = \int_a^x f(t)\,\mathrm{d}t \]
	% section das_unbestimmte_integral (end)


\end{document}
