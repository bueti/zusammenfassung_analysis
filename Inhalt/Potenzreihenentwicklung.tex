\chapter{Potenzreihenentwicklung}
\section{Unendliche Reihen}
\subsection{Grundbegriffe}
\subsubsection*{Unendliche Zahlenfolge}
\begin{definition}
Die Folge $\left<s_n\right>$ der Partialsummen einer unendlichen Zahlenfolge $\left<a_n\right>$ heisst \textit{unendliche Reihe}. Symbolische Schreibweise: 
$$ \sum\limits_{n=1}^{\infty} a_n = a_1 + a_2 + a_3 + ... + a_n + ..$$
\end{definition}

\subsubsection*{Konvergenz/Divergenz}
\begin{definition}Eine unendliche Reihe $\sum\limits_{n=1}^{\infty} a_n$ heisst \textit{konvergent} wenn die Folge ihrer Partialsummen $ s_n = \sum\limits_{k=1}^{n} a_k $ einen Grenzwert $s$ besitzt:
\[ \lim_{n \rightarrow \infty} s_n = \lim_{n \rightarrow \infty} \sum_{k=1}^{n} a_k = s \]
Dieser Grenzwert wird als \textit{Summenwert} der unendlichen Reihe bezeichnet. Symbolische Schreibweise:
$$\sum\limits_{n=1}^{\infty} a_n = a_1 + a_2 + a_3 + ...+ a_n +...=s$$
Besitzt die Partialsummenfolge $\left< s_n \right>$ jedoch \textit{keinen} Grenzwert, so ist die unendliche Reihe divergent.
\end{definition}

\subsection{Konvergenzkriterien}
\begin{definition}Für die \textit{Konvergenz} einer unendlichen Reihe $\sum\limits_{n=1}^{\infty} a_n$ ist die Bedingung 
$$ \lim\limits_{n \rightarrow \infty} a_n = 0$$
\textit{notwendig}, nicht aber hinreichend. Mit anderen Worten: Damit die unendliche Reihe \textit{konvergiert}, müssen die Reihenglieder diese Bedingung erfüllen. $\lim\limits_{n \rightarrow \infty} a_n = 0$ heisst aber keineswegs, das die unendliche Reihe konvergiert. Eine Reihe jedoch, die das notwendige Konvergenzkriterium nicht erfüllt, kann nicht konvergent sein und ist daher \textit{divergent}.
\end{definition}

\subsubsection*{Quotientenkriterium}
\begin{definition}
Erfüllen die Glieder einer unendlichen Reihe $\sum\limits_{n=1}^{\infty} a_n $ mit $ a_n \neq 0 $ alle $ n \in \mathbb{N}^*$ die Bedingung 
\[
\lim_{n \rightarrow \infty} \left| \frac{a_n+1}{a_n} \right|= q < 1
\]
so ist die Reihe \textit{konvergent}. Ist aber $q>1$, so ist die Reihe \textit{divergent}. 
\end{definition}

\subsubsection*{Wurzelkriterium}
\begin{definition}
Erfüllen die Glieder einer unendlichen Reihe $\sum\limits_{n=1}^{\infty} a_n $ die Bedingung 
\[
 \lim_{n \rightarrow \infty} \sqrt[n]{|a_n|} = q < 1
\]
so ist die Reihe \textit{konvergent}. 
\end{definition}

Für beide Kriterien gilt: Für \(q = 1\) versagt das Kriterium (keine Aussage möglich). Das Quotienten- oder Wurzelkriterium ist hinreichend aber nicht notwendig, d. h. es gibt Reihen, für die der Grenzwert nicht vorhanden ist aber trotzdem konvergieren.

\subsubsection*{Leibnizscheskriterium}
\begin{definition}Eine \textit{alternierende} Reihe vom Typ
$$ \sum\limits_{n=1}^{\infty} (-1)^{n+1} \cdot a_n = a_1 -1 a_2 + a_3 - a_4 + - ...$$
mit $a_n > 0$ für alle $n \in \mathbb{N}^*$ ist konvergent, wenn die Reihenglieder die folgenden Bedinungen erfüllen:
\begin{enumerate}
	\item $a_1 > a_2 > a_3 ...$ strikt monoton sinkend
	\item $\lim\limits_{n \rightarrow \infty} a_n = 0$
\end{enumerate}
\end{definition}

Beispiel:
$$ \sum\limits_{n=1}^{\infty} (-1)^{n+1} \cdot \frac{1}{n!}  = \frac{1}{1!} - \frac{1}{2!} + \frac{1}{3!} ... $$ 
$$ \lim\limits_{n\rightarrow \infty} \frac{1}{n!} = \lim\limits_{n\rightarrow \infty} \frac{1}{1 \cdot 2 \cdot 3 ...} = 0$$

\subsubsection*{Wichtige konvergente Reihen}
\begin{itemize}
	\item Geometrische Reihe: \\$\sum\limits_{n=1}^{\infty} aq^{n-1} = a + aq + aq^2 + ... + aq^{n-1} + ... = \frac{a}{1-q}$  $(|q| <1)$
	\item Harmonische Reihe: $\sum\limits_{n=1}^{\infty} \frac{1}{n} = 1 + \frac{1}{2} + \frac{1}{3} + \frac{1}{4} + ...$
	\item Alternierende harmonische Reihe: $\sum\limits_{n=1}^{\infty} \frac{(-1)^{n+1}}{n} = 1 - \frac{1}{2} + \frac{1}{3} - \frac{1}{4} + - ... = ln{2}$
	\item $1 + \frac{1}{1!} + \frac{1}{2!}+\frac{1}{3!} + ... = e$
\end{itemize}
Gut zu wissen:
\begin{itemize}
	\item $\frac{1}{(n+1)!} = \frac{1}{(n+1)n!}$
	\item $n! = n(n-1)(n-2)(...) = n(n-1)!$\\
	\item $\lim\limits_{n \rightarrow \infty}(1+\frac{1}{2})^n = e$
\end{itemize}

\subsection{Potenzreihen}
\begin{definition}Eine \textit{Potenzreihe} ist eine unendliche Reihe vom Typ $$P(x)= \sum\limits_{n=0}^{\infty} a_n x^n = a_0 + a_1x^1 + a_2 x^2 ...$$
Die reellen Zahlen $a_0, a_1, a_2, ...$ heissen \textit{Koeffizienten} der Potenzreihe. Der \textit{Konvergenzbereich} ist der gesamte Bereich in welchem die Reihe konvergiert. Der \textit{Konvergenzradius} ist eine Teilmenge des Konvergenzbereichs.
\end{definition}
Zu einer etwas allgemeineren Darstellungsform der Potenzreihen gelangt man durch die Definitionsvorschrift:
$$P(x) = \sum\limits_{n=0}^{\infty} a_n (x-x_0)^n = a_0 + a_1(x-x_0)^1 + a_2(x-x_0)^2 + ... + a_n(x-x_0)^n + ...$$
\begin{formel}
Der Konvergenzradius $r$ einer Potenzreihe lässit sich nach der Formel 
$$r = \lim\limits_{n \rightarrow \infty} \left|\frac{a_n}{a_n+1}\right|$$
$$r = \frac{1}{\lim\limits_{n \rightarrow \infty} \sqrt[n]{\left|a_n\right|}}$$
berechnen (Voraussetzung: alle Koeffizienten $a_n \neq 0$ und der Grenzwert ist vorhanden). Die Reihe konvergiert dann für $|x| < r$ und divergiert für $|x|>r$. In den beiden Randpunkten $x_1 = \pm r$ ist das Konvergenzverhalten der Potenzreihe zunächst unbestimmt.
\end{formel}

\subsubsection*{Berechnen des Konvergenzradius und Bestimmung des Verhaltens von $\pm r$}
\begin{enumerate}
	\item Konvergenzradius $r$ bestimmen
	\item Konvergenzverhalten in den Randpunkten bestimmen
\end{enumerate}

\section{Taylor-Reihen}
Die Taylorsche Formel sagt aus, dass man eine Funktion $f$ an der Stelle $x$ in der Umgebung eines bekannten Wertes $x_0$ als ein Polynom in $x$ schreiben kann, bis auf das sogenannte Lagrangsche Restglied. Das Restglied interessiert uns aber nicht wirklich.
\begin{formel}
$$f(x) = f(x_0) + \frac{f'(x_0)}{1!}(x-x_0) + \frac{f''(x_0)}{2!}(x-x_0)^2 + ... = \sum\limits_{n=0}^{\infty} \frac{f^{(n)}(x_0)}{n!}(x - x_0)^n$$
Die \textit{Mac Laurinsche Reihe} ist eine spezielle Form der Taylorschen Reihe für das Entwicklungszentrum $x_0 = 0$ (Nullpunkt):
$$f(x) = f(0) + \frac{f'(0)}{1!}x + \frac{f''(0)}{2!}x^2 + ... = \sum\limits_{n=0}^{\infty} \frac{f^{(n)}(0)}{n!}x^n$$
\end{formel}

\subsection{Beispiele der Taylorreihen}
\subsubsection*{Taylorreihe der Sinusfunktion}
Wir entwickeln die Sinusfunktion um die Stell $x_0 = \pi / 2$:
...
$$ \sum\limits_{n=0}^{\infty} (-1)^n \cdot \frac{(x-\pi / 2)^{2n}}{(2n)!}$$

\subsubsection*{Mac Laurin Reihe der Exponentialfunktion}
Wir bestimmen die Mac Laurinsche Reihe von $f(x) = e^x$:\\
$f(x) = f'(x) = f''(x) = ... = f^{(n)}(x) = ... = e^x$\\
$f(0) = f'(0) = f''(0) = ... = f^{(0)}(x) = ... = e^0 = 1$
$$e^x = 1 + \frac{x}{1!} + \frac{x^2}{2!} + ... + \frac{x^n}{n!} + ... = \sum\limits_{n=0}^{\infty} \frac{x^n}{n!}$$

\subsubsection*{Mac Laurin Reihe der Cosinusfunktion}
...
\begin{formel}
$$cos x = 1 - \frac{x^2}{2!} + \frac{x^4}{4!} - \frac{x^6}{6!} +- ... =  \sum\limits_{n=0}^{\infty}(-1)^n \cdot \frac{x^{2n}}{(2n)!}$$
\end{formel}

\subsubsection*{Mac Laurin Reihe der Sinusfunktion}
Die Mac Laurinsche Reihe der sinusfunktion erhalten wir am bequemsten durch gliedweise Differentiation der Kosinusreihe:
\\...\\
$$ = \sum\limits_{n=0}^{\infty} (-1)^n \cdot \frac{x^{2n+1}}{(2n+1)!}$$

\subsubsection*{Näherungspolynome der Kosinusfunktion}
Maybe... not!

\subsection{Grenzwertregel von Bernoulli und de L'Hopital}
\begin{definition}
Für Grenzwerte vom Typ $"\frac{0}{0}"$ bzw. $"\frac{\infty}{\infty}"$ gilt:
$$lim_{x \rightarrow x_0} \frac{f(x)}{g(x)} = lim_{x \rightarrow x_0} \frac{f'(x)}{g'(x)}$$
Anmerkungen:
\begin{itemize}
	\item Die Bernoulli-de L'Hospitalsche Regel setzt voraus, dass die Funktion $f(x)$ und $g(x)$ in der Umgebung von $x_0$ stetig differenzierbar sind und der Grenzwert der rechten Seite existiert.
	\item Die BdH Regel gilt sinngemäss auch für Grenzübergänge vom Typ $x \rightarrow \infty$ oder $x \rightarrow -\infty$.
	\item In einigen Fällen muss man mehrere Male Ableiten um zum Ziel zu kommen
\end{itemize}
\end{definition}

\subsubsection*{Nützliche Umformungen}
\renewcommand{\arraystretch}{3}
\begin{tabular}{|c|c|c|}
\hline 
\rule[-1ex]{0pt}{2.5ex} Funktion $\phi(x)$ & Grenzwert $lim_{x \rightarrow x_0} \phi(x)$& Elementare Umformung \\ 
\hline
\rule[-1ex]{0pt}{2.5ex} $u(x) \cdot v(x)$ & $0 \cdot \infty$ bzw. $\infty \cdot 0$ & $\frac{\frac{u(x)}{1}}{u(x)}$ bzw. $\frac{\frac{v(x)}{1}}{u(x)}$ \\ 
\hline 
\rule[-1ex]{0pt}{2.5ex} $u(x) - v(x)$ &$\infty - \infty$ & $\frac{\frac{1}{v(x)}-\frac{1}{u(x)}}{\frac{1}{u(x) \cdot v(x)}}$ \\ 
\hline 
\rule[-1ex]{0pt}{2.5ex} $u(x)^{v(x)}$ & $0^0$, $\infty^0$, $1^\infty$ & $e^{v(x) \cdot ln u(x)}$ \\ 
\hline 
\end{tabular} 