% Eine gute Codierung
\usepackage[utf8]{inputenc}
% Schriftart mit Umlauten
\usepackage[T1]{fontenc}
\usepackage{lmodern}
\usepackage{relsize} % Schriftgröße relativ festlegen
% Deutsche Beschriftung und Silbentrennung
\usepackage[ngerman]{babel}
% Für schöne Tabellen
\usepackage{longtable}
\usepackage{booktabs}
\usepackage{multirow}
\usepackage{array}
% Um Grafiken zu laden
\usepackage{graphicx}
\usepackage{float}
% Klickbare Indexes und Kosmetik
\usepackage{color}
\definecolor{black}{gray}{0} % 10% gray
\usepackage[colorlinks=true,linkcolor=black,citecolor=black]{hyperref}
% Untersrecihungen
\usepackage[normalem]{ulem}
\setcounter{secnumdepth}{5}
\setcounter{tocdepth}{1}
% Mathe
\usepackage{amsmath}
\usepackage{mathtools}
\usepackage{amsfonts}   
\usepackage{amssymb}
\DeclareFontFamily{U}{MnSymbolC}{}
\DeclareSymbolFont{MnSyC}{U}{MnSymbolC}{m}{n}
\DeclareFontShape{U}{MnSymbolC}{m}{n}{
    <-6>  MnSymbolC5
   <6-7>  MnSymbolC6
   <7-8>  MnSymbolC7
   <8-9>  MnSymbolC8
   <9-10> MnSymbolC9
  <10-12> MnSymbolC10
  <12->   MnSymbolC12%
}{}
\DeclareMathSymbol{\powerset}{\mathord}{MnSyC}{180}
% Anpassung Seitenlayout
\usepackage[
    automark, % Kapitelangaben in Kopfzeile automatisch erstellen
    headsepline, % Trennlinie unter Kopfzeile
    ilines % Trennlinie linksbündig ausrichten
]{scrpage2}
% Einfache Definition der Zeilenabstände und Seitenränder etc. -----------------
\usepackage{setspace}
\usepackage{geometry}