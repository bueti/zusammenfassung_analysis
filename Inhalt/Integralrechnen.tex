\chapter{Integralrechnen}
\section{Einführung} % (fold)
\label{sec:einführung}

\subsection{Definition} % (fold)
\label{sub:definition}
Die Integration ist die Umkehrung der Ableitung.
% subsection definition (end)
\subsection{Stammfunktion} % (fold)
\begin{enumerate}
	\item Falls es eine Stammfunktion zu einer stetigen Funktion \(f(x)\) gibt, gibt es \emph{unendlich} viele Stammfunktionen
	\item Zwei beliebige Stammfunktionen \(F_1 (x)\) und \(F_2 (x)\) von \(f(x)\) unterscheiden sich durch eine \emph{additive} Konstante: \(F_1 (x) - F_2 (x) = C\)
	\item Ist \(F_1 (x)\) eine \emph{beliebige} Stammfunktion von \(f (x)\), so ist auch \(F_1 (x) + C\) eine Stammfunktion von \(f (x)\). Daher lässt sich die \emph{Menge aller Stammfunktionen} in der Form \(F(x) = F_1 (x) + C \) darstellen.
\end{enumerate}
% section einführung (end)

\section{Unter- und Obersumme}
\[ \sum_{k=1}^{n} k = \frac{n(n+1)}{2} \]
\[ \sum_{k=1}^{n} k^2 = \frac{n(n+1)(2n+1)}{6} \]

\section{Das bestimmte Integral} % (fold)
\label{sub:das_bestimmte_integral}
Es gilt:\newline
\[A = \lim\limits_{n \rightarrow \infty}U_n = \lim\limits_{n \rightarrow \infty}O_n = \lim\limits_{n \rightarrow \infty}\sum_{k=1}^{n} f(x_k) \Delta x_k = \int_a^b f(x)\,\mathrm{d}x \]
% section das_bestimmte_integral (end)

\section{Das unbestimmte Integral} % (fold)
\label{sec:das_unbestimmte_integral}
\[ I(x) = \int_a^x f(t)\,\mathrm{d}t \]
% section das_unbestimmte_integral (end)

\section{Integration durch Substitution} % (fold)
\label{sec:integration_durch_substitution}
\begin{enumerate}
	\item Aufstellung der Substitutionsgleichung
	\item Durchführung der Integralsubstitution
	\item Integration
	\item Rücksubstitution
\end{enumerate}
% section integration_durch_substitution (end)

\section{Partielle Integration} % (fold)
\label{sec:partielle_integration}
\[ \int u \cdot v'\,\mathrm{d}x = u \cdot v - \int u' \cdot v\,\mathrm{d}x \]
\[ \int_a^b u \cdot v'\,\mathrm{d}x = [u \cdot v]_a^b - \int_a^b u' \cdot v\,\mathrm{d}x \]
Wenn der Integral-Minuend auf der rechten Seite gleich wie das Integral auf der linken Seite ist, kann es zur rechten Seite addiert werden und \emph{verschwindet} somit.
% section partielle_integration (end)

\section{Beispiele aus Physik und Technik} % (fold)
\label{sec:beispiele_aus_physik_und_technik}
Durch das Ableiten der Ortsfunktion \( s(t) \) erhalten wir die Momentan-Geschwindigkeit \(v(t)\) und durch nochmaliges Ableiten erhält man die Momentan-Beschleunigung \(a(t)\):
\[ v(t) = \frac { d }{ dt } s(t) = \dot { s } \]
\[ a(t) = \frac { d }{ dt } v(t) = \dot { v } = \ddot{s}\]
Umgekehrt gilt, falls \( a = a(t) \) bekannt ist, erhalten wir die Geschwindigkeit, bzw. den Ort durch Integration:
\[ v(t) = \int a(t)\,\mathrm{d}t \]
\[ s(t) = \int v(t)\,\mathrm{d}t = \iint a(t)\,\mathrm{d}t\]

% section beispiele_aus_physik_und_technik (end)

\section{Flächeninhalt} % (fold)
\label{sec:flächeninhalt}
\subsection{Bezgl. x-Achse} % (fold)
\label{sub:bezgl_x_achse}
\[ A = \left| \int_a^b f(x)\,\mathrm{d}x \right|\]
\[ A = A_1 + A_2 + ... A_n\]
\[ A = \left| \int_a^{x_1} f(x)\,\mathrm{d}x \right| + \left| \int_{x_1}^{x_2} f(x)\,\mathrm{d}x \right| + \left| \int_{x_{n-1}}^{x_n} f(x)\,\mathrm{d}x \right| ... \]
% subsection bezgl_x_achse (end)
\subsection{Zwischen zwei Kurven} % (fold)
\label{sub:zwischen_zwei_kurven}
\[ A = \int_a^b (y_o - y_u)\,\mathrm{d}x = \int_a^b [f_o(x) - f_u(x)]\,\mathrm{d}x\]
Dabei bedeuten:\newline
\( y_o = f_o(x)\): Gleichung der \emph{oberen} Randkurve\newline
\( y_u = f_u(x)\): Gleichung der \emph{unteren} Randkurve

Vorgehen:\newline
1) Kurvenschnittpunkte berechnen:\newline
\(f_o(x) = f_u(x)\) \newline
2) Integral nach obiger Formel berechnen.
% subsection zwischen_zwei_kurven (end)

% section flächeninhalt (end)

\section{Volumen eines Rotationkörpers} % (fold)
\label{sec:volumen_eines_rotationkörpers}
\subsection{Rotation um die x-Achse} % (fold)
\label{sub:rotation_um_die_x_achse}
\[ V_x = \pi \cdot \int_a^b y^2\,\mathrm{d}x = \pi \cdot \int_a^b [f(x)]^2\,\mathrm{d}x \]
% subsection rotation_um_die_x_achse (end)
\subsection{Rotation um die y-Achse} % (fold)
\label{sub:rotation_um_die_y_achse}
\[ V_x = \pi \cdot \int_c^d x^2\,\mathrm{d}y = \pi \cdot \int_c^d [g(y)]^2\,\mathrm{d}y \]
Die Grenzen liegen auf der y-Achse! Formel nach x auflösen.
% subsection rotation_um_die_y_achse (end)
% section volumen_eines_rotationkörpers (end)