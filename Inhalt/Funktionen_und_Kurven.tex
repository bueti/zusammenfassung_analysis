\chapter{Funktionen und Kurven}
\section{Darstellung} % (fold)
\label{sec:darstellung}

% section darstellung (end)
\subsection{Definition}
Unter einer Funktion versteht man eine Vorschrift, die jedem Element \(x\) aus einer Menge \(D\) genau ein Element \(y\) aus einer Menge \(M\) zuordnet.

\subsection{Darstellung}
Symbolische Schreibweise
\newline \(y = f(x)\) mit \(x \in D\) oder
\newline \(f: x \rightarrow y = f(x)\) mit \(x \in D\)

\subsection{Analytische Schreibweise}
Explizite Darstellung: \(y = f (x) \)
\newline Implizierte Darstellung: \(F(x;y) = 0\)

\subsection{Nullstellen} % (fold)
\label{sec:nullstellen}
Eine Funktion \(y = f(x)\) besitzt an der Stelle \(x_{0}\) eine Nullstelle, wenn \(f(x_{0}) = 0\) ist.
\newline In einer Nullstelle schneidet oder berührt die Funktionskurve die x-Achse!

\section{Symmetrie} % (fold)
\subsection{Gerade Funktionen} % (fold)
Eine Funktion \(y = f(x)\) mit einem zum Nullpunkt symmetrischen Definitionsbereich D heisst gerade, wenn für jedes \(x \in D\) gilt:
\newline \(f(-x) = f(x)\)

\subsection{Ungerade Funktionen}
Eine Funktion \(y = f(x)\) mit einem zum Nullpunkt symmetrischen Definitionsbereich D heisst ungerade, wenn für jedes \(x \in D\) gilt:
\newline \(f(-x) = -f(x)\)

\subsection{Beweis}
Um zu Beweisen ob eine Funktione gerade oder ungerade ist, setzt man -x in die Funktion ein.

\section{Monotonie}
\(x_{1}\) und \(x_{2}\) seien zwei beliebige Werte aus dem Definitionsbereich \(D\) einer Funktion \(y = f(x)\) mit \(x_{1} < x_{2}\). Dann heisst die Funktion:
\begin{itemize}
	\item monoton wachsend, falls \(f(x_{1}) \ge f(x_{2}))\)
	\item streng monoton wachsend, falls \(f(x_{1}) < f(x_{2}))\)
	\item monoton fallend, falls \(f(x_{1}) \le f(x_{2}))\)
	\item streng monoton fallend, falls \(f(x_{1}) > f(x_{2}))\)
\end{itemize}

\section{Periodizität}
Eine Funktion \(y = f(x)\) heisst periodisch mit Periode \(p\), wenn mit jedem \(x \in D\) auch \(x \pm p\) zum Definitionsbereich \(D\) der Funktion gehört und es gilt:
\newline \(f(x \pm p) = f(x)\)

\section{Umkehr/inverse Funktion}
Eine Funktion \(y = f(x)\) heisst umkehrbar, wenn aus \(x_{1} \neq x_{2}\) stets \(f(x_{1}) \neq f(x_{2})\) folgt. Sie muss streng monoton sein!
\par
Der Definitions- und Wertebereich wird bei einer Umkehrung getauscht:
\newline \(f^{-1}(f(x)) = x = f(f^{-1}(x))\)
\newline Bestimmung:
\begin{enumerate}
	\item nach \(x\) auflösen
	\item \(x\) und \(y\) tauschen
\end{enumerate}