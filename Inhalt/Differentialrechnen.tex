\chapter{Differentialrechnen}
\section{Differenzierbarkeit einer Funktion}
Eine Funktion \(y=f(x)\) heisst an der Stelle \(x_0\) differenzierbar, wenn der Grenzwert

\( \lim\limits_{\Delta x \rightarrow 0}{\frac{\Delta y}{\Delta x}} = \lim\limits_{\Delta x \rightarrow 0}{\frac{f(x_0 + \Delta x)-f(x_0)}{\Delta x}} = tan \alpha = m_{S(ekante)/T(angente)} \)

definiert ist. Man bezeichnet ihn als die (erste) Ableitung von \(y=f(x)\) an der Stelle \(x_0\) oder als \textbf{Differentialquotient} von \(y=f(x)\) an der Stelle \(x_0\) und kennzeichnet ihn durch das Symbol:

 \hspace*{10mm}\(y'(x_0)\), \(f'(x_0)\) oder \(\frac{dy}{dx} \mid_{x=x_0}\).

Weitere nützliche Schreibweise:

\hspace*{10mm}\(\frac{d}{dx}[f(x)] = f'(x)\)

\begin{itemize}
	\item Die Differenzierbarkeit einer Funktion \(y=f(x)\) an der Stelle \(x_0\) bedeutet, dass die Funktionskurve an dieser Stelle eine eindeutig bestimmte Tangente mit endlicher Steigung besitzt.
	\item Die Ableitungsfunktion \(y'(x)=f'(x)\) ordnet jeder Stelle \(x\) aus einem Intervall \(I\) als funktionswert den Steigungswert (Grenzwert) zu. Man spricht dann kurz von der Ableitung einer Funktion \(y=f(x)\) an der Stelle \(x\)
\end{itemize}
Eine Betragsfunktion ist an der Stelle \(x=0\) nicht differenzierbar, da sie dort keine eindeutig bestimmte Tangente besitzt. Man muss also zu erst die rechts- und dann die linksseitige Ableitung ausrechnen.
\par \textbf{Tangenten:} Für die Ableitung und den Steigungswinkel \(\alpha\) gilt \(y'=tan \alpha\)

\section{Ableitungen der elementaren Funktionen}

\begin{longtable}{|l|l|l|}
	\hline
	\multicolumn{2}{|l|}{Funktion f(x)}  & Ableitung f'(x) \\
	\hline
	Konstante Funktion & c = const & 0 \\
	\hline
	Potenzfunktion & \(x^n\) \newline \((x^2)^{-\frac{1}{2}} \) & \(n \cdot x^{n-1}\) \newline \(\frac{1}{(x^{2})^{\frac{1}{2}}}\) \\
	\hline
	Wurzelfunktion & \(\sqrt{x}\) gilt nur für 2. Wurzel & \(\frac{1}{2 \sqrt{x}}\) \\
	\hline
	\multirow{4}{*}{Trigonometrische Funktionen} & \(sin x\) \newline \(sin 2x\) & \(cos x\) \newline \(2 cos (2x)\) \\ \cline{2-3}
	& \(cos(x)\)  & \(-sin x\) \\ \cline{2-3}
	& \(tan(x)\) & \(\frac{1}{cos^2(x)}\), \( 1 + tan^2(x) \) \\ \cline{2-3}
	& \(cot(x)\) & \(-\frac{1}{sin^2(x)}\) \\
	\hline
	\multirow{4}{*}{Arkusfunktionen} & \(arcsin(x)\) & \(\frac{1}{\sqrt{1 - x^2}} \) \\ \cline{2-3}
	& \(arccos(x)\) & \(-\frac{1}{\sqrt{1-x^2}}\) \\ \cline{2-3}
	& \(arctan(x)\) & \(\frac{1}{1+x^2}\) \\ \cline{2-3}
	& \(arccot(x)\) & \(-\frac{1}{1+x^2}\) \\
	\hline
	\multirow{3}{*}{Exponentialfunktionen} & \(e^x\) & \( e^x \) \\ \cline{2-3}
	& \(e^{2x}\) & \( 2e^{2x} \) \\ \cline{2-3}
	& \(a^x\) & \((ln a) \cdot a^x \) \\ \hline
	\multirow{2}{*}{Logarithmusfunktionen} & \(ln x\) & \( \frac{1}{x} \) \\ \cline{2-3}
	& \(log_a x\) & \( \frac{1}{(ln a) \cdot x } \) \\ \hline
	\multicolumn{3}{|l|}{Hyperbelfunktionen, Areafunktionen: Siehe Seite 17 im Skript.} \\ \hline
\end{longtable}

\section{Ableitungsregeln}
\subsection{Faktorregel}
\(y = C \cdot f(x) \Rightarrow y' = C \cdot f'(x)\); C = Reelle Konstante, bleibt erhalten.
\subsection{Summenregel}
\(y = f_1(x)+f_2(x)+ ... +f_n(x) \Rightarrow y'=f'_1(x) + f'_2(x)+...+f'_n(x)\)

Es darf gliedweise differenziert werden.
\subsection{Produktregel}
\(y=u \cdot v \Rightarrow y'=u' \cdot v + v' \cdot u\)

Bei mehreren Termen:

\(y=u \cdot v \cdot w \Rightarrow y'=u' \cdot v \cdot w + u \cdot v' \cdot w + u \cdot v \cdot w' \)
\subsection{Quotientenregel}
\(y = \frac{u}{v} \Rightarrow y' = \frac{u' \cdot v - v' \cdot u}{v^2}\)

\section{Kettenregel}
Überlegung: Was muss zuerst berechnet werde, was gehört zusammen?

Beispiel:

\hspace*{10mm} \(f(x) = arccos \sqrt{x^2-1} \)
\newline
\begin{tabular}{|l|l|}
	\hline
   	innerste Funktion z & \(z: x \rightarrow z(x) = x^2-1\) \\
	\hline
	mittlere Funktion u & \(u: z \rightarrow u(z) = \sqrt{z}\) \\
	\hline
	äusserste Funktion F & \(F: u \rightarrow F(u) = arccos(u) \) \\
	\hline
 \end{tabular}

\section{Tangente und Normale}
\begin{longtable}{p{0.5\textwidth}|p{0.5\textwidth}}
	Steigung der Tangente & \(m_T=f'(x)\)\\
	\hline
	Gleichung der Tangente & \(y_T = f'(x_0)(x-x_0) + y_0 \)\\
	\hline
	Steigung der Normalen & \(m_N=-\frac{1}{f'(x_0)} = \frac{y-y_0}{x-x_0} \)\\
	\hline
	Gleichung der Normalen & \(y_N = - \frac{1}{f'(x_0)}(x-x_0) + y_0 \) \\
\end{longtable}

\section{Linearisierung einer Funktion}
In der Umgebung des Kurvenpunkts \(P=(x_0,y_0)\) kann die nichtlineare Funktion \(y=f(x)\) näherungsweise durch die lineare Funktion\newline
\(y - y_0 = f'(x_0) \cdot (x - x_0) \) oder \( \Delta y = f'(x_0)\Delta x \)\newline
ersetzt werden.

\par Vorgehen: 
\begin{enumerate}
	\item Arbeitspunkt P bestimmen: \(x_0, y_0\)
	\item Tangentensteigung:
	\begin{enumerate}
		\item erste Ableitung
		\item \(m_T = f'(x_0)\)
	\end{enumerate}
	\item Gleichung der Tangente in P
	\begin{enumerate}
		\item \(\frac{y-y_0}{x-x_0} = m_T  \Rightarrow y = 2x-3\pi  \)
	\end{enumerate}
	\item Im Arbeitspunkt P gibt linearisierte Funktion
	\item Exakter Wert: Ursprungsfunktion
	\item Näherungswert: Linearisierte Funktion
\end{enumerate}

\section{Monotonie einer Kurve} % (fold)
\label{sec:monotonie_einer_kurve}
\( y' = f'(x) > 0 \Rightarrow \) streng monoton wachsend\newline
\( y' = f'(x) < 0 \Rightarrow \) streng monoton fallend
% section monotonie_einer_kurve (end)

\section{Krümmung einer Kurve} % (fold)
\label{sec:krümmung_einer_kurve}
\subsection{Allgemein} % (fold)
\label{sub:allgemein}
\( y'' = f''(x_0) > 0 \): Links-Krümmung. \( f'(x_0) \) muss streng monoton wachsend sein. \newline
\( y'' = f''(x_0) < 0 \): Rechts-Krümmung. \( f'(x_0) \) muss streng monoton fallend sein. \newline
Formel zur Berechnung der Krümmung einer ebenen Kurve \(y=f(x)\):\par
\( \kappa = \kappa (x) = \frac{y''}{[1+(y')^2]^{\frac{3}{2}}} = \frac{f''(x)}{[1+[f'(x)]^2]^{\frac{3}{2}}} \)
% subsection allgemein (end)
% section krümmung_einer_kurve (end)
\section{Charakterische Kurvenpunkte} % (fold)
\label{sec:charakterische_kurvenpunkte}
\subsection{Extremum} % (fold)
\label{sub:extremum}
Eine Funktion \(y = f(x) \) besitzt an der Stelle \(x_0\) ein relatives Maximum bzw. ein relatives Minimum, wenn in einer gewissen Umgebung von \(x_0\) stets \newline
\(f(x_0) > f(x) \) bzw. \( f(x_0) < f(x) \) \newline
ist (\(x \neq x_0\)).
	\[f'(x_0) = 0 \wedge f''(x_0) \neq 0 \]

Für \(f''(x_0) > 0 \) liegt dabei eine relatives Minimum vor, für \(f''(x_0) < 0 \) dagegen ein relatives Maximum.
% subsection extremum (end)
\subsection{Wendepunkte, Sattelpunkte} % (fold)
% \label{sub:wendepunkte}
Punkte \(x_0\) mit \(f''(x_0) = 0 \) und \(f'''(x_0) \neq 0 \) heissen \textbf{Wendepunkte} (die Krümmung wechselt ihre Richtung).
\par Kurven mit waagrechter Tangente werden als Sattelpunkte bezeichnet.
\par Hinreichende Bedingung für einen \textbf{Wendepunkt}: \newline
\[f''(x_0) = 0 \] und \[ f'''(x_0) \neq 0 \]
\par Notwendige Bedingung für einen \textbf{Sattelpunkt}: \newline
\[f'(x_0) = 0 \wedge f''(x_0) = 0 \wedge f'''(x_0) \neq 0 \]
% subsection wendepunkte (end)

% section charakterische_kurvenpunkte (end)
\section{Kurvendisskusion}
Ablauf:\newline
\begin{itemize}
	\item Definitionsbereich/lücken
	\item Symmetrie (gerade, ungerade Funktion)
	\item Nullstellen, Schnittpunkt mit der y-Achse
	\item Pole, senkrechte Asymptoten (Polgeraden)
	\item Ableitungen (i. d. R. bis zur 3. Ordnung)
	\item Relative Extremwerte (Maxima, Minima)
	\item Wendepunkte, Sattelpunkte
	\item Verhalten der Funktion für \(x \rightarrow \pm \infty \), Asymptoten im Unendlichen
	\item Wertebereich der Funktion
	\item Zeichnung der Funktion in einem geeigneten Massstab
\end{itemize}